\documentclass[12pt, a4paper] {article}
\usepackage[utf8] {inputenc}
\usepackage[T2A]{fontenc}
\usepackage[english, russian] {babel}
\usepackage[usenames,dvipsnames]{xcolor}
\usepackage{minted,longtable,amsmath,amsfonts,graphicx,tikz,hyperref}
\usepackage{indentfirst,verbatim}

\hypersetup{
    colorlinks,
    linkcolor={red!50!black},
    citecolor={blue!50!black},
    urlcolor={blue!80!black}
}

\setminted{
  fontsize=\scriptsize,
  baselinestretch=1.2,
  linenos,
  frame=lines
}

\begin{document}
\thispagestyle{empty}
\begin{center}
  {\large
    Университет ИТМО \\
    Кафедра Информатики и прикладной математики \\
  }
\end{center}
\vspace{\stretch{2}}
\begin{center}
  {\large
    Машинное обучение\\
  }
    \vspace{\stretch{1}}
  {\large
    Лабораторная работа 3\\
    ``Методы дискриминантного анализа''\\
  }
\end{center}
\vspace{\stretch{6}}
\begin{flushright}
  Работу выполнил студент группы P4117\\
  {\it Фомин Евгений\\
  }
\end{flushright}
\vspace{\stretch{4}}
\begin{center}
  2017
\end{center}
\newpage

\section{Постановка задачи}

\begin{enumerate}
    \item Прочитать теоретическую часть по методам дискриминантного анализа.
    \item Осуществить визуализацию двух любых признаков и посчитать коэффициент
    корреляции между ними
    \item Выполнить разбиение классов набора данных с помощью LDA
    (LinearDiscriminantAnalysis). Осуществить визуализацию разбиения
    \item Осуществить классификацию с помощью методов LDA и QDA
    (LinearDiscriminantAnalysis и QuadraticDiscriminantAnalysis). Сравнить
    полученные результаты.
\end{enumerate}

\section{Исходные данные}
\subsection{Датасет}
\begin{description}
  \item[Источник] \href{https://archive.ics.uci.edu/ml/datasets/Covertype}{https://archive.ics.uci.edu/ml/datasets/Covertype}
  \item[Предметная область] Тип лесного покрова
  \item[Задача]Определить, к какому из 7 типов относится лесной покров конкретного участка земли
  \item[Количество записей] 10000
  \item[Количество атрибутов] 54
\end{description}
\subsection{Атрибуты}
\begin{description}
  \item[Elevation] Высота (число)
  \item[Aspect] Сторона (число)
  \item[Slope] Угол склона (число)
  \item[Horizontal distance to hydrology] Горизонтальная составляющая вектора направления к ближайшему водоему (число)
  \item[Vertical distance to hydrology] Вертикальная составляющая вектора направления к ближайшему водоему (число)
  \item[Hillshade 9 am] Hillshade 9 am (число)
  \item[Hillshade Noon] Hillshade Noon (число)
  \item[Hillshade 3 pm] Hillshade 3 pm (число)
  \item[Horizontal distance to fire points] Дистанция до ближайших точек лесных пожаров (число)
  \item[Wilderness Area] Дикость (4 бинарных поля)
  \item[Soil type] Типы почвы (40 бинарных полей)
\end{description}

\section{Ход работы}
\inputminted{python}{listing.py}

\section{Реализация за пределами машинного обучения}
Для визуализации данных было решено применить веб-технологии, в частности,
библиотеку \textit{d3}. Графическая составляющая написана на языке Clojurescript
с применением библиотеки Reagent и вызовами к d3. По результатам работы выяснилось,
что для простых графиков эффективнее применить библиотеки менее широкого профиля,
нежели Data Driven Documents.

Минимальный сервер для вызова приведенного в листинге кода написан с помощью
Flask.

\end{document}
