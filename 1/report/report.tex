\documentclass[12pt, a4paper] {article}
\usepackage[utf8] {inputenc}
\usepackage[T2A]{fontenc}
\usepackage[english, russian] {babel}
\usepackage[usenames,dvipsnames]{xcolor}
\usepackage{minted,longtable,amsmath,amsfonts,graphicx,tikz,hyperref}
\usepackage{indentfirst,verbatim}

\hypersetup{
    colorlinks,
    linkcolor={red!50!black},
    citecolor={blue!50!black},
    urlcolor={blue!80!black}
}

\setminted{
  fontsize=\scriptsize,
  baselinestretch=1.2,
  linenos,
  frame=lines
}

\begin{document}
\thispagestyle{empty}
\begin{center}
  {\large
    Университет ИТМО \\
    Кафедра Информатики и прикладной математики \\
  }
\end{center}
\vspace{\stretch{2}}
\begin{center}
  {\large
    Машинное обучение\\
  }
    \vspace{\stretch{1}}
  {\large
    Лабораторная работа 1\\
    ``Метрические алгоритмы классификации''\\
  }
\end{center}
\vspace{\stretch{6}}
\begin{flushright}
  Работу выполнил студент группы P4117\\
  {\it Фомин Евгений\\
  }
\end{flushright}
\vspace{\stretch{4}}
\begin{center}
  2017
\end{center}
\newpage

\section{Постановка задачи}

\begin{enumerate}
  \item Программно реализовать два метрических алгоритма классификации:
    Naive Bayes и K Nearest Neighbours
  \item Сравнить работу реализованных алгоритмов с библиотечными из scikit-learn
  \item Для тренировки, теста и валидации использовать один из предложенных датасетов
  \item Сформировать краткий отчет
\end{enumerate}

\section{Исходные данные}
\subsection{Датасет}
\begin{description}
  \item[Источник] \href{https://archive.ics.uci.edu/ml/datasets/Covertype}{https://archive.ics.uci.edu/ml/datasets/Covertype}
  \item[Предметная область] Тип лесного покрова
  \item[Задача]Определить, к какому из 7 типов относится лесной покров конкретного участка земли
  \item[Количество записей] 10000
  \item[Количество атрибутов] 54
\end{description}
\subsection{Атрибуты}
\begin{description}
  \item[Elevation] Высота (число)
  \item[Aspect] Сторона (число)
  \item[Slope] Угол склона (число)
  \item[Horizontal distance to hydrology] Горизонтальная составляющая вектора направления к ближайшему водоему (число)
  \item[Vertical distance to hydrology] Вертикальная составляющая вектора направления к ближайшему водоему (число)
  \item[Hillshade 9 am] Hillshade 9 am (число)
  \item[Hillshade Noon] Hillshade Noon (число)
  \item[Hillshade 3 pm] Hillshade 3 pm (число)
  \item[Horizontal distance to fire points] Дистанция до ближайших точек лесных пожаров (число)
  \item[Wilderness Area] Дикость (4 бинарных поля)
  \item[Soil type] Типы почвы (40 бинарных полей)
\end{description}

\section{Ход работы}
\subsection{Naive Bayes}
\inputminted{clojure}{bayes.clj}
\subsection{KNN}
\inputminted{clojure}{knn.clj}
\subsection{Сравнение с библиотечными функциями}
\inputminted{python}{listing.py}

Основным отличием в реализациях наивного байесовского классификатора, полученной в ходе
выполнения лабораторной работы, и библиотечной является различие атрибутов количественных и
качественных: для первых вероятность считается по закону распределения Гаусса, а для вторых
как дискретное распределение со сглаживанием Лапласа. В результате это позволило повысить точность.

\end{document}
